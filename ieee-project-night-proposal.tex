%\documentclass[article]{memoir}
\documentclass{article}
\usepackage[hidelinks]{hyperref}

%\counterwithout{section}{chapter}
%\maxtocdepth{subsection}
%\setcounter{tocdepth}{2}
%\setlength{\cftsubsectionindent}{5em}


\begin{document}

\title{IEEE Project Night: The Details}
\author{
Alex Hirzel\thanks{\href{mailto:alex@hirzel.us}{\texttt{alex@hirzel.us}}},
IEEE E-Board\thanks{(names go here)}}
\maketitle

\begin{abstract}
The IEEE Project Night is a weekly event put on by the IEEE Lab to encourage
students to work on personal projects. Every Project Night is catered and
students are encouraged to start new projects and finish existing ones. This
directly benefits students and campus as a whole by providing ample opportunity
for teamwork and collaboration in an open environment.
\end{abstract}

\tableofcontents

\newpage


\section{Background}

Imagine you're a bright mind at Tech.\footnote{You may not need to imagine.} You
have a lot of great ideas and maybe some half-done projects that are stalled.
You started most of them during breaks, over the summer, on that free weekend
back in September. You intend to finish them some day, but right now it seems
like there's no time for that. You're cramped on one side by school, you're
working a part-time job, you have a social life---there's just never a good
time, so it never happens.

Despite all this, you and your friends worked on your projects just the other
night. You made it happen. You participated in a Hack-A-Thon. You allocated the
time to do the work and planned around it in advance. Everybody agreed on the
time, and everybody was there. It felt good to finish a project!

\begin{itemize}
\item The IEEE Project Night is a wake-up call: you probably have everything you
need to finish your project except the impetus and the time slot.
\item The IEEE Project Night gets you working on these projects and is a great
benefit to campus---your work reflects on everyone!
\end{itemize}

Several IEEE members recognized that the recent hack-a-thon (TechHacks 2013)
provided the impetus needed to start and complete projects. The reason for this
was the structure and organization of the event and the fact that it was
scheduled---it could be planned-around! The IEEE lab is organizing this event in
the spirit of a recurring hack-a-thon. It will become the de-facto work time for
members and non-members alike to work on personal projects.


\subsection*{The appeal of this event is structure and free food}

Yes, that deserves its own header. TechHacks was a success because it brought
everyone together at a unified time.



\section{Five-Year Vision}



\section{Goals}

asdf


\subsection{Qualitative}

asdf


\subsection{Quanitative}

asdf



\section{Benefits}

Participants bring their projects and a desire to work on them.


\subsection{For all students}

alskjgla sdlfa dsjlfa kds


\subsection{For the IEEE Student Organization}

alskjgla sdlfa dsjlfa kds


\subsection{For Michigan Tech}

alskjgla sdlfa dsjlfa kds



\section{Schedule}

\subsection{Spring 2014 (kick-off)}

\subsection{AY 2014-2015}

\subsection{Infinity and Beyond}
The general recurrence relation for this event will depend 


\section{Funding Opportunities}

\end{document}

