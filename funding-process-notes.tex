\documentclass[article]{memoir}
\usepackage[superscript]{cite}
\usepackage[hidelinks]{hyperref}

\counterwithout{section}{chapter}
\newcommand{\bi}[0]{\begin{itemize}}
\newcommand{\ei}[0]{\end{itemize}}
\begin{document}

\section{Undergraduate Student Government}

The Undergraduate Student Government will take applications against the
Opportunities Fund, which "provide[s] funding for startup costs of new
organizations; unpredicted expenses that promote the student organization, or
new initiatives of student organizations that were not previously budgeted for"
\cite{mtuusg}. \textbf{Unfortunately, an application for funding to cater the
Project Night is likely to be denied}, because USG "will not fund food giveaways
to members or non-members"\cite{usgsite}. Nonwithstanding, the loose process to
get funding for mid-semester unbudgeted expenses via this mechanism is as
follows \cite{usgsite}:

\bi
\item Pick up \emph{Opportunity Request Form} from MUB 106
\item Submit form and schedule meeting with \emph{Ways and Means} committee
\item Give 3-5 minute pitch in a USG meeting
	\bi
	\item USG and Ways and Means combined decision determines fund amount
	\ei
\item Standard reimbursement or advanced reimbursement
\ei



\section{Wellness}

The HOWL\footnote{Healthy Options for a Wellness Lifestyle} organization under
Wellness will fund "alcohol-alternative" events.\cite{wellness} Getting this
funding is an open-ended process, and it will be best to schedule Project Nights
concurrently with popular part times (i.e. Friday and Saturday nights). This
will appeal maximally to HOWL's objectives for the funding. The process of
getting this funding is simply to email
\href{mailto:wellness@mtu.edu}{\texttt{wellness@mtu.edu}}. Below is an example
message that could be sent to open this door.

\begin{quotation}
To Whom It May Concern,

My name is Alex Hirzel and I am writing in representation of the IEEE student
chapter here on campus. We are a group of nerds and geeks, and we're trying to
organize a recurring event in Spring 2014 that could be beneficial for both of
our organizations. Toward the end of this message, I will propose a funding
relationship that will align with our goals (build a strong member base,
increase involvement in our group) and yours as well (promote a recurring,
well-advertised, destructive-decision-displacing event that is free and open to
all students). I look forward to hearing your thoughts on this after I dive in
and explain.

Firstly, the IEEE is \emph{the} international career organization for electrical
and computer engineers; hundreds of thousands of members strong, the IEEE is a
prime mover in the field. The IEEE lab on campus and the IEEE Student group are
linked to this national organization.

The IEEE Student group is principally concerned with ...

Many of us students are busy with classes and homework, but we also have
personal projects. Example of some of these projects are given later (there are
a \emph{lot} of projects and ideas floating around!) A consensus of members was
recently formed and we decided we would like to explore an event where, weekly,
we could gather and work on our projects as a group. There are many up-sides to
this for us, including:

\bi
\item Collaboration opportunities. There are a lot of members with niche
knowledge---our one or two microcontroller experts and our resident high voltage
afficianados as examples---and it really benefits the "younger" members to have
them around.
\item Prioritization of extracurricular work. Engineering majors with project
experience are much stronger applicants to jobs than those who just have strong
academic credentials. Michigan Tech as a whole would benefit from more students
participating in extracurricular engineering projects.
\ei

It is our vision that we can "scratch two itches" with this event. By scheduling
it on nights where alcohol consumption is usually high (e.g. Friday or Saturday
night), we can hope to snag a few people who would otherwise be under the
influence. To me, this is a socially-optimal outcome and an excellent use of
the funds you make available for this exact purpose.

The form of support we believe will serve us best is catering and advertising
support for our event. To be explicit, we envision your organization donating
pizza (less than \$50 per event) and helping us to promote.

I have attached a paper detailing our desired schedule for the events as well as
more information on what will actually take place during them. Peruse it at your
convenience, or feel free to contact us with questions.

If you would like to meet to discuss this, please let me know your preferred
avenue (be it email, your regularly-scheduled meeting, our regularly-scheduled
meeting, carrier pigeon, etc.) and I will ensure we're able to talk. I very much
look forward to your reply. Happy holidays!

Alex

\end{quotation}



\begin{thebibliography}{9}

\bibitem{mtuusg} Michigan Tech Student Organizations - Funding Sources\newline
{\footnotesize\url{http://www.sa.mtu.edu/stulife/stuorg\%20copy/funding_sources/index.html}}

\bibitem{usgsite} Undergraduate Student Government - Finance\newline
{\footnotesize\url{http://usg.mtu.edu/usg/home/finance-ways-and-means}}

\bibitem{wellness} Student Activities - Wellness\newline
{\footnotesize\url{http://www.mtu.edu/student-activities/oap-wellness/wellness/}}

\end{thebibliography}


\end{document}
